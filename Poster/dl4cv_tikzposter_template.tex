\documentclass[25pt, a0paper, landscape]{tikzposter}
\tikzposterlatexaffectionproofoff
\usepackage[utf8]{inputenc}
\usepackage{authblk}
\makeatletter
\renewcommand\maketitle{\AB@maketitle} % revert \maketitle to its old definition
\renewcommand\AB@affilsepx{\quad\protect\Affilfont} % put affiliations into one line
\makeatother
\renewcommand\Affilfont{\Large} % set font for affiliations
\usepackage{amsmath, amsfonts, amssymb}
\usepackage{tikz}
\usepackage{pgfplots}
% align columns of tikzposter; needs two compilations
\usepackage[colalign]{column_aligned}

% tikzposter meta settings
\usetheme{Default}
\usetitlestyle{Default}
\useblockstyle{Default}

%%%%%%%%%%% redefine title matter to include one logo on each side of the title; adjust with \LogoSep
\makeatletter
\newcommand\insertlogoi[2][]{\def\@insertlogoi{\includegraphics[#1]{#2}}}
\newcommand\insertlogoii[2][]{\def\@insertlogoii{\includegraphics[#1]{#2}}}
\newlength\LogoSep
\setlength\LogoSep{-70pt}

\renewcommand\maketitle[1][]{  % #1 keys
    \normalsize
    \setkeys{title}{#1}
    % Title dummy to get title height
    \node[inner sep=\TP@titleinnersep, line width=\TP@titlelinewidth, anchor=north, minimum width=\TP@visibletextwidth-2\TP@titleinnersep]
    (TP@title) at ($(0, 0.5\textheight-\TP@titletotopverticalspace)$) {\parbox{\TP@titlewidth-2\TP@titleinnersep}{\TP@maketitle}};
    \draw let \p1 = ($(TP@title.north)-(TP@title.south)$) in node {
        \setlength{\TP@titleheight}{\y1}
        \setlength{\titleheight}{\y1}
        \global\TP@titleheight=\TP@titleheight
        \global\titleheight=\titleheight
    };

    % Compute title position
    \setlength{\titleposleft}{-0.5\titlewidth}
    \setlength{\titleposright}{\titleposleft+\titlewidth}
    \setlength{\titlepostop}{0.5\textheight-\TP@titletotopverticalspace}
    \setlength{\titleposbottom}{\titlepostop-\titleheight}

    % Title style (background)
    \TP@titlestyle

    % Title node
    \node[inner sep=\TP@titleinnersep, line width=\TP@titlelinewidth, anchor=north, minimum width=\TP@visibletextwidth-2\TP@titleinnersep]
    at (0,0.5\textheight-\TP@titletotopverticalspace)
    (title)
    {\parbox{\TP@titlewidth-2\TP@titleinnersep}{\TP@maketitle}};

    \node[inner sep=0pt,anchor=west] 
    at ([xshift=-\LogoSep]title.west)
    {\@insertlogoi};

    \node[inner sep=0pt,anchor=east] 
    at ([xshift=\LogoSep]title.east)
    {\@insertlogoii};

    % Settings for blocks
    \normalsize
    \setlength{\TP@blocktop}{\titleposbottom-\TP@titletoblockverticalspace}
}
\makeatother
%%%%%%%%%%%%%%%%%%%%%%%%%%%%%%%%%%%%%


% color handling
\definecolor{TumBlue}{cmyk}{1,0.43,0,0}
\colorlet{blocktitlebgcolor}{TumBlue}
\colorlet{backgroundcolor}{white}

% title matter
\title{DL4CV Project}

\author[1]{Asif, Paul, Theo, Tom}

%\affil[1]{Technical University of Munich}

\insertlogoi[width=15cm]{tum_logo}
\insertlogoii[width=15cm]{tum_logo}

% main document
\begin{document}

\maketitle




\begin{columns}
	
	\column{0.3}
	\block{Segmantic Segmentation}{
		\begin{itemize}	
			\item What are we interested in
			\item How do we want to segment it
			\item What should be the outcome
			\item What is the expected quality
		\end{itemize}
	}

	\block{Udacity Simulator Environment}{
		\begin{itemize}	
			\item Short description of how it works
			\item Why did we choose it over GTA(Linux compatibility issues)
			\item 
			\item 
		\end{itemize}
	}

	\column{0.4}

	\block{Cityscapes Dataset}{
	\begin{itemize}	
		\item Sceenshots
		\item Size
		\item Features
	\end{itemize}
}
	\block{Segmentation Architecture}{Content in your block.}
	\block{Udacity Dataset}{
		\begin{itemize}	
		\item Sceenshots
		\item Size
		\item Features
	\end{itemize}
	}
	\block{Nvidia Architecture}{
		\begin{itemize}	
		\item Sceenshot of the architecure
		\item What we changed
		\item Blackbox analysis(Whats the input whats the output)
	\end{itemize}
	}

	\column{0.3}
	\block{Training Results}{Leave space for glueing later on the results}
	\block{Testing Results}{Leave space for glueing later on the results}

	\block{Training Results}{Leave space for glueing later on the results}
	\block{Testing Results}{Leave space for glueing later on the results}
\end{columns}

\block{References}

\end{document}